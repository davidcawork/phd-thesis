\chapter{Repositorios públicos}
\label{ch:repositorios}

Con el objetivo de garantizar la reproducibilidad científica y fomentar la transferencia de resultados a la comunidad investigadora, todos los desarrollos realizados en esta Tesis Doctoral se han liberado como software de acceso abierto. En esta sección se recopilan los repositorios públicos asociados a cada una de las contribuciones principales, incluyendo código fuente, scripts de experimentación y documentación técnica.

\begin{itemize}
    \item \textbf{Extensión win-BOFUSS para control \textit{in-band}}: implementación del switch de referencia con soporte para control ligero en escenarios IoT. Disponible en: \url{https://github.com/NETSERV-UAH/in-BOFUSS}.
    
    \item \textbf{Algoritmo DEN2NE}: propuesta escalable para distribución y reasignación automática de recursos mediante etiquetado jerárquico. Disponible en: \url{https://github.com/NETSERV-UAH/den2ne-Alg}.
    
    \item \textbf{Predicción y resiliencia en \gls{sg}}: prototipos de predicción de fallos y reconfiguración proactiva en redes eléctricas inteligentes, con conjuntos de datos y scripts de entrenamiento. Disponible en: \url{https://github.com/PaulaBartolomeMora/TFM}.
    
    \item \textbf{Algoritmo BLOSTE}: algoritmo multicriterio para el encaminamiento adaptativo en redes de distribución malladas. Disponible en: \url{https://github.com/NETSERV-UAH/den2ne-SmartGrids}.
    
    \item \textbf{Arquitectura \textit{data-driven} para \gls{iiot}}: sistema contenerizado de monitorización, inferencia y control, acompañado de datasets abiertos y guías de despliegue. Disponible en: \url{https://github.com/NETSERV-UAH/datadriven-poc}.
\end{itemize}

La liberación de estos repositorios no solo aporta transparencia y validez a los resultados presentados, sino que también ofrece a otros investigadores una base práctica para extender, replicar y validar los mecanismos desarrollados en este trabajo.