%%%%%%%%%%%% Frase celebre %%%%%%%%%%%%

% Limpiamos el estilo de la página
\thispagestyle{empty}

\vspace*{4cm}

\begin{flushright}
\itshape
``No os preguntarán por mí,\\
que en estos tiempos a nadie le da lustre\\
haber nacido segundón en casa grande;\\
pero si pregunta alguno,\\
bueno será contestarle\\
que, español, a toda vena,\\
amé, reñí, di mi sangre,\\
pensé poco, recé mucho,\\
jugué bien, perdí bastante.\\[0.7em]

Y, porque esa empresa loca que nunca debió tentarme,\\
que, perdiendo ofende a todos,\\
que, triunfando alcanza a nadie,\\
no quise salir del mundo\\
sin poner mi pica en Flandes.\\[0.7em]

¡Por España!\\[0.7em]

Y el que quiera defenderla honrado muera.\\
Y el traidor que la abandone,\\
no tenga quien le perdone,\\
ni en Tierra Santa cobijo,\\
ni una cruz en sus despojos,\\
ni las manos de un buen hijo,\\
para cerrarle los ojos.''
\end{flushright}

\vspace{1em}
\begin{flushright}
\textsc{Hernando de Acuña}
\end{flushright}

%%%%%%%%%%%%%%%%%%%%%%%%%%%%%%%%%
\clearemptydoublepage %salta a nueva página impar

%%%%%%%%%%%% Funding %%%%%%%%%%%%

% Limpiamos el estilo de la página
\thispagestyle{empty}

\vspace*{\fill}

\begin{flushleft}
\begin{minipage}{\textwidth}
%\raggedright
\scriptsize
Este trabajo ha sido parcialmente financiado por la Universidad de Alcalá a través del programa FPI-UAH 2022; por la Comunidad de Madrid mediante los proyectos TAPIR-CM (S2018/TCS-4496), MistLETOE-CM (CM/JIN/2021-006), TUCAN6-CM (TEC-2024/COM-460) y VERANO (CM/DEMG/2024-038); por el MICIU y la Unión Europea NextGenerationEU/PRTR a través del proyecto ADMINISTER (TED2021-131301B-I00); por el Ministerio de Ciencia e Innovación mediante el proyecto ONENESS (PID2020-116361RA-I00); y por el Ministerio de Asuntos Económicos y Transformación Digital y la Unión Europea–NextGenerationEU a través del proyecto UNICO 5G I+D 6G-DATADRIVEN-02, coordinado por la Universidad Carlos III de Madrid.
\end{minipage}
\end{flushleft}

%%%% Funding

% +   Comunidad de Madrid
%     -   TAPIR-CM (S2018/TCS-4496)
%     -   MistLETOE-CM (CM/JIN/2021-006)
%     -   TUCAN6-CM (TEC-2024/COM-460)
%     -   VERANO (CM/DEMG/2024-038)

% +   MICIU y la Unión Europea NextGenerationEU/PRTR
%     -   ADMINISTER (TED2021-131301B-I00)

% +   Spanish Ministry of Science and Innovation 
%     -   ONENESS (PID2020-116361RA-I00)

% +   Spanish Ministry of Economic Affairs and Digital Transformation and the European Union-NextGenerationEU
%     -   UNICO 5G I+D 6G-DATADRIVEN-02 project coordinated by Universidad Carlos III de Madrid.

% +   Este trabajo ha sido parcialmente financiado por la Universidad de Alcalá a través del programa FPI-UAH 2022;

\vspace{1cm}

%%%%%%%%%%%%%%%%%%%%%%%%%%%%%%%%%
\clearemptydoublepage %salta a nueva página impar

%%%%%%%%%%%% Agradecimientos %%%%%%%%%%%%
\chapter*{Agradecimientos}

% Limpiamos el estilo de la página
%\thispagestyle{empty}

\vspace{1cm}

Es increíble cómo pasa el tiempo, y da vértigo echar la vista atrás. Pensar en cómo he llegado hasta aquí, las conferencias a las que he tenido la suerte de asistir, los viajes que he podido realizar, y, sobre todo, en las personas que he conocido por el camino. A menudo se describe el doctorado como un proceso arduo y exigente, y no diré lo contrario, pero, también merece la pena poner en valor todo lo aprendido, lo vivido y las experiencias que me han hecho crecer tanto a nivel personal como profesional.\\
\\
Escribo estas líneas desde Milán, durante mi estancia doctoral junto a Marco Savi, a quien quiero agradecer sinceramente todo el tiempo, dedicación y esfuerzo que me ha brindado. Su apoyo ha hecho que esta etapa sea verdaderamente inolvidable. Esta experiencia no habría sido la misma sin todas las personas maravillosas que he conocido en Italia, que me han hecho sentir como en casa, siempre con una sonrisa y dispuestas a ayudar.\\
\\
Volviendo la mirada a casa, no puedo dejar de agradecer a mis amigos, que han estado siempre a mi lado, apoyándome en cada paso. A mi familia, por su paciencia infinita, por su ayuda constante y por soportarme en los momentos de estrés y agobio de esta ``empresa loca'' que es el doctorado. A mis amigos del LE-34, los que están y los que ya no están, con quienes he compartido tantos cafés, confidencias, comidas y tardes de ``trabajo''. Y, por supuesto, a todos mis compañeros del grupo NetIS, que tras tantos años, más que colegas se han convertido en una segunda familia. Gracias por creer en mí y por acompañarme hasta aquí.\\
\\
Por último, a Elisa y Diego, mis directores de tesis. Gracias por vuestra guía, vuestro apoyo incondicional y por confiar en mí desde el primer momento. En especial a Elisa: si no fuera por aquel correo que me enviaste hace ya más de siete años, ofreciéndome una mini beca de investigación, hoy no estaría escribiendo estas líneas. Gracias por todo lo que me habéis enseñado, por lo que he aprendido con vosotros y por todo lo que aún me queda por aprender.

\vspace{0.5cm}

Sinceramente, mil gracias a todos.



%%%%%%%%%%%%%%%%%%%%%%%%%%%%%%%%%
\clearemptydoublepage %salta a nueva página impar


%%%%%%%%%%%% Resumen %%%%%%%%%%%%

\chapter{Resumen}

La era contemporánea en la cual vivimos se caracteriza en mayor medida por una profunda transformación tecnológica y social, impulsada por la globalización y por el desarrollo de infraestructuras digitales interconectadas que configuran lo que se conoce como el Internet of Everything (IoE). En este nuevo paradigma, emergen redes densas y altamente heterogéneas en las que convergen dispositivos, servicios y plataformas con requerimientos funcionales muy variopintos, integrando no solo redes de comunicaciones, sino también infraestructuras energéticas, industriales y logísticas. Esta complejidad creciente demanda nuevas metodologías para un control y gestión, que sea en la medida de lo posible, flexible y escalable. En este contexto, las redes softwarizadas y programables se postulan como una elemento tecnológico clave, al permitir una abstracción funcional de la infraestructura subyacente, facilitar su automatización y promover la integración de capacidades de control, así como, la adaptación dinámica de la red, a las necesidades intrinsecas de los nodos de la misma.\\
\\
Esta Tesis contribuye al desarrollo de las redes softwarizadas y programables mediante la propuesta de soluciones orientadas a mejorar la gestión, la resiliencia y la cooperación entre nodos de dichas redes. En primer lugar, se diseñan y evalúan algoritmos de toma de decisiones inteligentes en entornos altamente dinámicos y heterogéneos, con aplicación en dominios como el Internet de las Cosas Industrial (IIoT), las redes eléctricas inteligentes (Smart Grids) y las redes de comunicaciones de nueva generación. Estas soluciones permiten una asignación dinámica de recursos, una adaptación proactiva a las condiciones del entorno y una reducción sustancial en la complejidad de gestión de la red. Además, se ha abordado la integración de modelos de Inteligencia Artificial (AI) con dichos algoritmos, con el fin de potenciar la detección temprana de fallos, lo que se traduce en una mejora significativa de la resiliencia, y la alta disponibilidad de los servicios. En segundo lugar, se propone una arquitectura software modular, orientada a servicios y alineada con estándares actuales, que permite la incorporación de herramientas emergentes, así como mecanimos automatizados con AI, ofreciendo capacidades  de computación tanto en la nube y como en el edge. Esta arquitectura está concebida para proporcionar una infraestructura lógica robusta, interoperable y escalable, capaz de facilitar la orquestación autónoma de servicios distribuidos, orientado a contextos de elevada heterogeneidad tecnológica, como entornos IIoT u otros.

\vspace{0.5cm}

\textbf{Palabras clave}: Redes densas y hetereogeneas, Redes programables y softwarizadas, Algoritmos, Infraestructura Cloud, IoT industrial, Smart grids.


%%%%%%%%%%%%%%%%%%%%%%%%%%%%%%%%%
\clearemptydoublepage %salta a nueva página impar


%%%%%%%%%%%% Abstract %%%%%%%%%%%%
\chapter{Abstract}

The contemporary era is marked by a profound technological and social transformation, driven by globalization and the pervasive deployment of interconnected digital infrastructures that define the Internet of Everything (IoE). Within this emerging paradigm, dense and highly heterogeneous networks are formed, comprising a wide variety of devices, services, and platforms with diverse and demanding functional requirements. These systems integrate not only communication networks, but also energy, industrial, and logistics infrastructures. The increasing complexity of such environments necessitates the adoption of novel methodologies capable of ensuring the flexible and scalable control and management of distributed resources and services. In this context, softwarized and programmable networks have emerged as a pivotal technological solution, enabling functional abstraction of the underlying infrastructure, supporting automation, and fostering the integration of advanced control mechanisms, as well as the dynamic adaptation of network behavior to the intrinsic requirements of its nodes.\\
\\
This Thesis contributes to the advancement of softwarized and programmable networking by proposing solutions designed to enhance the management, resilience, and cooperation between nodes within these infrastructures. Firstly, it presents and evaluates intelligent decision-making algorithms tailored for highly dynamic and heterogeneous environments, with applications in domains such as the Industrial Internet of Things (IIoT), smart grids, and next-generation communication networks. These algorithms facilitate dynamic resource allocation, proactive environmental adaptation, and a significant reduction in network management complexity. Furthermore, the integration of Artificial Intelligence (AI) models into these solutions is explored to enable early fault detection, thus improving system resilience and ensuring high service availability. Secondly, the Thesis proposes a modular, service-oriented software architecture, aligned with current standards and capable of incorporating emerging technologies and AI-driven automation mechanisms. This architecture offers computing capabilities across both cloud and edge infrastructures and is designed to deliver a robust, interoperable, and scalable logical platform that supports the autonomous orchestration of distributed services in technologically heterogeneous scenarios such as IIoT environments.

\vspace{0.5cm}

\textbf{Keywords}: Dense and heterogeneous networks, Programmable and softwarized networks, Algorithms, Cloud infrastructure, Industrial IoT, Smart Grids.

%%%%%%%%%%%%%%%%%%%%%%%%%%%%%%%%%
\clearemptydoublepage %salta a nueva página impar