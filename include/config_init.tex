% Define una geometría de página cuidada y adecuada para impresión
\usepackage[DIV=12,BCOR=12mm,headinclude=true,footinclude=false]{typearea}

%Para corregir los margenes de paginas pares e impares para impresion a dos caras. Sin esto salen al reves.
\let\tmp\oddsidemargin
\let\oddsidemargin\evensidemargin
\let\evensidemargin\tmp
\reversemarginpar

%Para agrandar un poco el margen superior y empequeñecer el inferior, que me parecia estaban un poco descompensados
\voffset=0.5cm

% Añade al índice y numera hasta la profundidad 4.
% 1:section,2:subsection,3:subsubsection,4:paragraph
\setcounter{tocdepth}{4}
\setcounter{secnumdepth}{4}

%%%%%%%%%%%%%%%%%%%%%%%%
% BIBLIOGRAFÍA
%%%%%%%%%%%%%%%%%%%%%%%%
\usepackage[numbers]{natbib}
\usepackage{breakcites,notoccite}

%%%%%%%%%%%%%%%%%%%%%%%%
% DOCUMENTO EN ESPAÑOL
%%%%%%%%%%%%%%%%%%%%%%%%
\usepackage[spanish]{babel}
\decimalpoint
\usepackage[utf8]{inputenc}
\usepackage[T1]{fontenc}

%%%%%%%%%%%%%%%%%%%%%%%% 
% COLORES
%%%%%%%%%%%%%%%%%%%%%%%% 
% Biblioteca de colores
\usepackage{color}
\usepackage[table,xcdraw,dvipsnames]{xcolor}
% Otros colores definidos por el usuario
\definecolor{gray97}{gray}{.97}
\definecolor{gray75}{gray}{.75}
\definecolor{gray45}{gray}{.45}
\definecolor{gray30}{gray}{.30}
\definecolor{negro}{RGB}{0,0,0}
\definecolor{blanco}{RGB}{255,255,255}
\definecolor{dkgreen}{rgb}{0,.6,0}
\definecolor{dkblue}{rgb}{0,0,.6}
\definecolor{dkyellow}{cmyk}{0,0,.8,.3}
\definecolor{gray}{rgb}{0.5,0.5,0.5}
\definecolor{mauve}{rgb}{0.58,0,0.82}
\definecolor{deepblue}{rgb}{0,0,0.5}
\definecolor{deepred}{rgb}{0.6,0,0}
\definecolor{deepgreen}{rgb}{0,0.5,0}
\definecolor{MyDarkGreen}{rgb}{0.0,0.4,0.0}
\definecolor{bluekeywords}{rgb}{0.13,0.13,1}
\definecolor{greencomments}{rgb}{0,0.5,0}
\definecolor{redstrings}{rgb}{0.9,0,0}
\definecolor{pantone293}{RGB}{35,91,168}

%%%%%%%%%%%%%%%%%%%%%%%%
% TABLAS
%%%%%%%%%%%%%%%%%%%%%%%%
% Paquetes para tablas
\usepackage{longtable,booktabs,array,multirow,multicol,tabularx,ragged2e,array}
\usepackage{graphicx}

% Nuevos tipos de columna para tabla, se pueden utilizar como por ejemplo C{3cm} en la definición de columnas de la función tabular
\newcolumntype{L}[1]{>{\raggedright\let\newline\\\arraybackslash\hspace{0pt}}m{#1}}
\newcolumntype{C}[1]{>{\centering\let\newline\\\arraybackslash\hspace{0pt}}m{#1}}
\newcolumntype{R}[1]{>{\raggedleft\let\newline\\\arraybackslash\hspace{0pt}}m{#1}}

%%%%%%%%%%%%%%%%%%%%%%%% 
% GRAFICAS y DIAGRAMAS 
%%%%%%%%%%%%%%%%%%%%%%%% 
% Paquete para todo tipo de gráficas, diagramas, modificación de imágenes, etc
\usepackage{rotating}
\usepackage{tikz,tikzpagenodes}
\usetikzlibrary{tikzmark,calc,shapes.geometric,arrows, arrows.meta,backgrounds,shadings,shapes.arrows,shapes.symbols,shadows,positioning,fit,automata,patterns,intersections}


\usepackage{pgfplots}
\pgfplotsset{colormap/jet}
\pgfplotsset{compat=newest} % Compatibilidad
\usepgfplotslibrary{patchplots,groupplots,fillbetween,polar}
\usepackage{pgfplotstable}

%%%%%%%%%%%%%%%%%%%%%%%% 
% FIGURAS, TABLAS, ETC 
%%%%%%%%%%%%%%%%%%%%%%%% 
\usepackage{subcaption} % Para poder realizar subfiguras
\usepackage{caption} % Para aumentar las opciones de diseño
% Nombres de figuras, tablas, etc, en negrita la numeración, todo con letra small
\captionsetup{labelfont={bf,small},textfont=small}
% Paquete para modificar los espacios arriba y abajo de una figura o tabla
\usepackage{setspace}
% Define el espacio tanto arriba como abajo de las figuras, tablas
\setlength{\intextsep}{5mm}
% Para ajustar tamaños de texto de toda una tabla o grafica
% Uso: {\scalefont{0.8} \begin{...} \end{...} }
\usepackage{scalefnt}


%%%%%%%%%%%%%%%%%%%%%%%% 
% OTROS
%%%%%%%%%%%%%%%%%%%%%%%%
% Para hacer una pagina horizontal. Uso: \begin{landscape} xxxx \end{lanscape}
\usepackage{lscape}
% Para incluir paginas PDF. Uso:
% \includepdf[pages={1}]{tuarchivo.pdf}
\usepackage{pdfpages}
% Para introducir url's con formato. Uso: \url{http://www.google.es}
\usepackage{url}
% Paquete que añade el hipervinculo en referencias dentro del documento, indice, etc
% Se define sin bordes alrededor. Uso: \ref{tulabel}
\usepackage[pdfborder={000}]{hyperref}
\usepackage{float}
\usepackage{placeins}
\usepackage{afterpage}
\usepackage{verbatim}

%%%%%%%%%%%%%%%%%%%%%%%% 
% GLOSARIOS
%%%%%%%%%%%%%%%%%%%%%%%%
\usepackage[acronym,nonumberlist,toc]{glossaries}
\usepackage{glossary-superragged}
\newglossarystyle{modsuper}{%
  \setglossarystyle{super}%
  \renewcommand{\glsgroupskip}{}
}
\renewcommand{\glsnamefont}[1]{\textbf{#1}}


%%%%%%%%%%%%%%%%%%%%%%%% 
% MATEMÁTICAS Y ALGORITMOS
%%%%%%%%%%%%%%%%%%%%%%%%
\usepackage{mathtools,amsthm,amsfonts,amssymb,bm,mathrsfs,nicefrac,upgreek,bigints}
\usepackage[linesnumbered,ruled,vlined,spanish]{algorithm2e}

%%%%%
% PARÁMETROS DE FORMATO DE CODIGOS
%%%%%
% Puedes editar los formatos para ajustarlos a tu gusto
\input{include/styles_prog.tex}


%%%%%%%%%%%%%%%%%%%%%%%%%%
% Frase celebre
%%%%%%%%%%%%%%%%%%%%%%%%%%
\newenvironment{FraseCelebre}
{\begin{list}{}{%
      \setlength{\leftmargin}{0.5\textwidth}
      % Desplazamos el inicio de
      % los párrafos a la derecha la mitad
      % de la anchura de la línea de texto.
      % Puede que quieras cambiar esto
      % por otra cantidad como '5cm'.
      \setlength{\parsep}{0cm}
      % La separación entre párrafos de la
      % frase o de la fuente es normal, sin
      % espacio extra.
      \addtolength{\topsep}{0.5cm}
      % Aumentamos un poco la separación
      % entre la parte de la fase célebre
      % y los párrafos de alrededor
    }
  }
  {\unskip \end{list}}

\newenvironment{Frase}%
{\item \begin{flushright}\small\em}%
  {\end{flushright}}

\newenvironment{Fuente}%
{\item \begin{flushright}\small}%
  {\end{flushright}}

\newenvironment{bottomparagraph}{\par\vspace*{\fill}}{\clearpage}

% Para que las páginas en blanco no tengan numeración.
\newcommand{\clearemptydoublepage}{\newpage{\pagestyle{empty}\cleardoublepage}}

%%%%%%%%%%%%%%%%%%%%%%%%%%%%
% Configuración del español extra
%%%%%%%%%%%%%%%%%%%%%%%%%%%%
\addto\captionsspanish{%
  \renewcommand{\listtablename}{Índice de tablas}
  \renewcommand{\tablename}{Tabla}
  \renewcommand{\lstlistingname}{Código}
  \renewcommand{\lstlistlistingname}{Índice de Códigos}
  \renewcommand{\glossaryname}{Glosario}
  \renewcommand{\acronymname}{Acrónimos}
  \renewcommand{\bibname}{Bibliografía}
}

%%%%%%%%%%%%%%%%%%%%%%%%%%%%%%
% Encabezados y demás
%%%%%%%%%%%%%%%%%%%%%%%%%%%%%%
\usepackage{fancyhdr}
\pagestyle{fancy}

\fancyhf{} % limpia encabezado y pie

% Encabezado izquierdo (páginas pares): capítulo
\fancyhead[LE]{\scshape \leftmark}

% Encabezado derecho (páginas impares): sección
\fancyhead[RO]{\scshape \rightmark}

% Línea horizontal en el encabezado
\renewcommand{\headrulewidth}{0.2pt}

% Número de página centrado en el pie
\fancyfoot[C]{\thepage}

% Redefinir \chaptermark para que \leftmark muestre "Capítulo N. Título"
\renewcommand{\chaptermark}[1]{%
  \markboth{\chaptername\ \thechapter.\ #1}{}}

% Redefinir \sectionmark para que \rightmark muestre "N.M Sección"
\renewcommand{\sectionmark}[1]{%
  \markright{\thesection\ #1}}

% Paquete para los forest de diego paper de den2ne + eca + aut
\usepackage{forest}