% Define una geometría de página cuidada y adecuada para impresión
\usepackage[DIV=12,BCOR=12mm,headinclude=true,footinclude=false]{typearea}

%Para corregir los margenes de paginas pares e impares para impresion a dos caras. Sin esto salen al reves.
\let\tmp\oddsidemargin
\let\oddsidemargin\evensidemargin
\let\evensidemargin\tmp
\reversemarginpar

%Para agrandar un poco el margen superior y empequeñecer el inferior, que me parecia estaban un poco descompensados
\voffset=0.5cm

% Añade al índice y numera hasta la profundidad 4.
% 1:section,2:subsection,3:subsubsection,4:paragraph
\setcounter{tocdepth}{4}
\setcounter{secnumdepth}{4}

%%%%%%%%%%%%%%%%%%%%%%%%
% BIBLIOGRAFÍA
%%%%%%%%%%%%%%%%%%%%%%%%
\usepackage[numbers]{natbib}
\usepackage{breakcites,notoccite}

%%%%%%%%%%%%%%%%%%%%%%%%
% DOCUMENTO EN ESPAÑOL
%%%%%%%%%%%%%%%%%%%%%%%%
\usepackage[spanish]{babel}
\usepackage[utf8]{inputenc}
\usepackage[T1]{fontenc}

%%%%%%%%%%%%%%%%%%%%%%%% 
% COLORES
%%%%%%%%%%%%%%%%%%%%%%%% 
% Biblioteca de colores
\usepackage{color}
\usepackage[table,xcdraw,dvipsnames]{xcolor}
% Otros colores definidos por el usuario
\definecolor{gray97}{gray}{.97}
\definecolor{gray75}{gray}{.75}
\definecolor{gray45}{gray}{.45}
\definecolor{gray30}{gray}{.30}
\definecolor{negro}{RGB}{0,0,0}
\definecolor{blanco}{RGB}{255,255,255}
\definecolor{dkgreen}{rgb}{0,.6,0}
\definecolor{dkblue}{rgb}{0,0,.6}
\definecolor{dkyellow}{cmyk}{0,0,.8,.3}
\definecolor{gray}{rgb}{0.5,0.5,0.5}
\definecolor{mauve}{rgb}{0.58,0,0.82}
\definecolor{deepblue}{rgb}{0,0,0.5}
\definecolor{deepred}{rgb}{0.6,0,0}
\definecolor{deepgreen}{rgb}{0,0.5,0}
\definecolor{MyDarkGreen}{rgb}{0.0,0.4,0.0}
\definecolor{bluekeywords}{rgb}{0.13,0.13,1}
\definecolor{greencomments}{rgb}{0,0.5,0}
\definecolor{redstrings}{rgb}{0.9,0,0}
\definecolor{pantone293}{RGB}{35,91,168}

%%%%%%%%%%%%%%%%%%%%%%%%
% TABLAS
%%%%%%%%%%%%%%%%%%%%%%%%
% Paquetes para tablas
\usepackage{longtable,booktabs,array,multirow,multicol,tabularx,ragged2e,array}
\usepackage{graphicx}

% Nuevos tipos de columna para tabla, se pueden utilizar como por ejemplo C{3cm} en la definición de columnas de la función tabular
\newcolumntype{L}[1]{>{\raggedright\let\newline\\\arraybackslash\hspace{0pt}}m{#1}}
\newcolumntype{C}[1]{>{\centering\let\newline\\\arraybackslash\hspace{0pt}}m{#1}}
\newcolumntype{R}[1]{>{\raggedleft\let\newline\\\arraybackslash\hspace{0pt}}m{#1}}

%%%%%%%%%%%%%%%%%%%%%%%% 
% GRAFICAS y DIAGRAMAS 
%%%%%%%%%%%%%%%%%%%%%%%% 
% Paquete para todo tipo de gráficas, diagramas, modificación de imágenes, etc
\usepackage{rotating}
\usepackage{tikz,tikzpagenodes}
\usetikzlibrary{tikzmark,calc,shapes.geometric,arrows, arrows.meta,backgrounds,shadings,shapes.arrows,shapes.symbols,shadows,positioning,fit,automata,patterns,intersections}


\usepackage{pgfplots}
\pgfplotsset{colormap/jet}
\pgfplotsset{compat=newest} % Compatibilidad
\usepgfplotslibrary{patchplots,groupplots,fillbetween,polar}
\usepackage{pgfplotstable}

%%%%%%%%%%%%%%%%%%%%%%%% 
% FIGURAS, TABLAS, ETC 
%%%%%%%%%%%%%%%%%%%%%%%% 
\usepackage{subcaption} % Para poder realizar subfiguras
\usepackage{caption} % Para aumentar las opciones de diseño
% Nombres de figuras, tablas, etc, en negrita la numeración, todo con letra small
\captionsetup{labelfont={bf,small},textfont=small}
% Paquete para modificar los espacios arriba y abajo de una figura o tabla
\usepackage{setspace}
% Define el espacio tanto arriba como abajo de las figuras, tablas
\setlength{\intextsep}{5mm}
% Para ajustar tamaños de texto de toda una tabla o grafica
% Uso: {\scalefont{0.8} \begin{...} \end{...} }
\usepackage{scalefnt}


%%%%%%%%%%%%%%%%%%%%%%%% 
% OTROS
%%%%%%%%%%%%%%%%%%%%%%%%
% Para hacer una pagina horizontal. Uso: \begin{landscape} xxxx \end{lanscape}
\usepackage{lscape}
% Para incluir paginas PDF. Uso:
% \includepdf[pages={1}]{tuarchivo.pdf}
\usepackage{pdfpages}
% Para introducir url's con formato. Uso: \url{http://www.google.es}
\usepackage{url}
% Paquete que añade el hipervinculo en referencias dentro del documento, indice, etc
% Se define sin bordes alrededor. Uso: \ref{tulabel}
\usepackage[pdfborder={000}]{hyperref}
\usepackage{float}
\usepackage{placeins}
\usepackage{afterpage}
\usepackage{verbatim}

%%%%%%%%%%%%%%%%%%%%%%%% 
% GLOSARIOS
%%%%%%%%%%%%%%%%%%%%%%%%
\usepackage[acronym,nonumberlist,toc]{glossaries}
\usepackage{glossary-superragged}
\newglossarystyle{modsuper}{%
  \setglossarystyle{super}%
  \renewcommand{\glsgroupskip}{}
}
\renewcommand{\glsnamefont}[1]{\textbf{#1}}


%%%%%%%%%%%%%%%%%%%%%%%% 
% MATEMÁTICAS Y ALGORITMOS
%%%%%%%%%%%%%%%%%%%%%%%%
\usepackage{mathtools,amsthm,amsfonts,amssymb,bm,mathrsfs,nicefrac,upgreek,bigints}
\usepackage[linesnumbered,ruled,vlined,spanish]{algorithm2e}

%%%%%
% PARÁMETROS DE FORMATO DE CODIGOS
%%%%%
% Puedes editar los formatos para ajustarlos a tu gusto
%%%%%%%%%%%%%%%%%%%%%%%% 
% CÓDIGO. CONFIGURACIÓN. En el siguiente bloque están los estilos.
%%%%%%%%%%%%%%%%%%%%%%%%
% Paquete para mostrar código de matlab. En caja y lineas numeradas
\usepackage[framed,numbered]{matlab-prettifier}
% Paquete mostrar código de programación de distintos lenguajes

\usepackage{listings}
\lstset{ inputencoding=utf8,
extendedchars=true,
frame=single, % Caja donde se ubica el código
backgroundcolor=\color{gray97}, % Color del fondo de la caja
rulesepcolor=\color{black},
boxpos=c,
abovecaptionskip=-4pt,
aboveskip=12pt,
belowskip=0pt,
lineskip=0pt,
framerule=0pt,
framextopmargin=4pt,
framexbottommargin=4pt,
framexleftmargin=11pt,
framexrightmargin=0pt,
linewidth=\linewidth,
xleftmargin=\parindent,
framesep=0pt,
rulesep=.4pt,
stringstyle=\ttfamily,
showstringspaces = false,
showspaces = false,
showtabs = false,
columns=fullflexible,
basicstyle=\small\ttfamily,
commentstyle=\color{gray45},
keywordstyle=\bfseries,
tabsize=4,
numbers=left,
numbersep=1pt,
numberstyle=\tiny\ttfamily\color{gray75},
numberfirstline = false,
breaklines=true,
postbreak=\mbox{\textcolor{red}{$\hookrightarrow$}\space}, % Flecha al saltar de linea
prebreak=\mbox{\textcolor{red}{$\hookleftarrow$}\space}, % Flecha al saltar de linea
literate=
  {á}{{\'a}}1 {é}{{\'e}}1 {í}{{\'i}}1 {ó}{{\'o}}1 {ú}{{\'u}}1
  {Á}{{\'A}}1 {É}{{\'E}}1 {Í}{{\'I}}1 {Ó}{{\'O}}1 {Ú}{{\'U}}1
  {à}{{\`a}}1 {è}{{\`e}}1 {ì}{{\`i}}1 {ò}{{\`o}}1 {ù}{{\`u}}1
  {À}{{\`A}}1 {È}{{\'E}}1 {Ì}{{\`I}}1 {Ò}{{\`O}}1 {Ù}{{\`U}}1
  {ä}{{\"a}}1 {ë}{{\"e}}1 {ï}{{\"i}}1 {ö}{{\"o}}1 {ü}{{\"u}}1
  {Ä}{{\"A}}1 {Ë}{{\"E}}1 {Ï}{{\"I}}1 {Ö}{{\"O}}1 {Ü}{{\"U}}1
  {â}{{\^a}}1 {ê}{{\^e}}1 {î}{{\^i}}1 {ô}{{\^o}}1 {û}{{\^u}}1
  {Â}{{\^A}}1 {Ê}{{\^E}}1 {Î}{{\^I}}1 {Ô}{{\^O}}1 {Û}{{\^U}}1
  {œ}{{\oe}}1 {Œ}{{\OE}}1 {æ}{{\ae}}1 {Æ}{{\AE}}1 {ß}{{\ss}}1
  {ű}{{\H{u}}}1 {Ű}{{\H{U}}}1 {ő}{{\H{o}}}1 {Ő}{{\H{O}}}1
  {ç}{{\c c}}1 {Ç}{{\c C}}1 {ø}{{\o}}1 {å}{{\r a}}1 {Å}{{\r A}}1
  {€}{{\euro}}1 {£}{{\pounds}}1 {«}{{\guillemotleft}}1
  {»}{{\guillemotright}}1 {ñ}{{\~n}}1 {Ñ}{{\~N}}1 {¿}{{?`}}1,
  }

% Intenta no dividir los códigos en diferentes paginas si es posible
\lstnewenvironment{listing}[1][]
   {\lstset{#1}\pagebreak[0]}{\pagebreak[0]}

% Formato de títulos de los códigos
\DeclareCaptionFont{white}{\color{white}}
\DeclareCaptionFormat{listing}{\colorbox{gray}{\parbox{\textwidth - 2\fboxsep}{#1#2#3}}}
\captionsetup[lstlisting]{format=listing,labelfont=white,textfont=white,font= scriptsize}


%%%%%%%%%%%%%%%%%%%%%%%% 
% CÓDIGO. ESTILOS. Ajústalos a tu gusto
%%%%%%%%%%%%%%%%%%%%%%%%
\DeclareOldFontCommand{\bf}{\normalfont\bfseries}{\mathbf}
\lstdefinestyle{Consola}
	{
	basicstyle=\scriptsize\bf\ttfamily,
	showstringspaces=false,
    commentstyle=\color{deepgreen},
    keywordstyle=\color{blue},
	}

\lstdefinestyle{Consola2}
	{
	basicstyle=\footnotesize\bf\ttfamily,
	showstringspaces=false,
    commentstyle=\color{deepgreen},
    keywordstyle=\color{blue},
	}
	
\lstdefinestyle{C}
	{
	basicstyle=\scriptsize,
	language=C,
	}
\lstdefinestyle{C-color}
	{
  	breaklines=true,
  	language=C,
  	basicstyle=\scriptsize,
  	keywordstyle=\bfseries\color{green!40!black},
  	commentstyle=\itshape\color{purple!40!black},
  	identifierstyle=\color{blue},
  	stringstyle=\color{orange},
    }
\lstdefinestyle{P4-color}
	{
  	breaklines=true,
  	language=C,
  	basicstyle=\scriptsize,
  	keywordstyle=\bfseries\color{green!40!black},
  	commentstyle=\itshape\color{purple!40!black},
  	identifierstyle=\color{blue},
  	stringstyle=\color{orange},
  	morekeywords={%
      action, algorithm, apply, attributes, bytes,
      calculated_field, control, counter, current, default, direct,
      drop, else, false, field_list, field_list_calculation, fields,
      header, header_type, hit, if, input, instance_count, last, latest,
      layout, length, mask, max_length, metadata, meter, min_width, miss
      output_width, packets, parse_error, parser, parser_exception, payload,
      primitive_action, register, result, return, saturating, select,
      signed, static, switch, true, type, update, valid, verify, width}
    }

\lstdefinestyle{CSharp}
	{
	basicstyle=\scriptsize
	language=[Sharp]C,
	escapeinside={(*@}{@*)},
	keywordstyle=\bfseries,
	}
\lstdefinestyle{CSharp-color}
	{
	basicstyle=\scriptsize
	language=[Sharp]C,
	escapeinside={(*@}{@*)},
	commentstyle=\color{greencomments},
	keywordstyle=\color{bluekeywords}\bfseries,
	stringstyle=\color{redstrings},
	}
\lstdefinestyle{C++}
	{
	basicstyle=\scriptsize,
	language=C++,
 	}
 	
\lstdefinestyle{C++-color}
	{
  	breaklines=true,
  	language=C++,
  	basicstyle=\scriptsize,
  	keywordstyle=\bfseries\color{green!40!black},
  	commentstyle=\itshape\color{purple!40!black},
  	identifierstyle=\color{blue},
  	stringstyle=\color{orange},
    }
    
\lstdefinestyle{PHP}
	{
	basicstyle=\scriptsize,
	language=PHP,
	}
	
\lstdefinestyle{PHP-color}
	{
	basicstyle=\scriptsize,
	language=PHP,
	keywordstyle    = \color{dkblue},
  	stringstyle     = \color{red},
  	identifierstyle = \color{dkgreen},
  	commentstyle    = \color{gray},
  	emph            =[1]{php},
  	emphstyle       =[1]\color{black},
  	emph            =[2]{if,and,or,else},
  	emphstyle       =[2]\color{dkyellow}
  }
  
\lstdefinestyle{Matlab}
	{
	basicstyle=\scriptsize,
	language=Matlab,
	numberstyle=\tiny\ttfamily\color{gray75},
	}
	
\lstdefinestyle{Matlab-color}
	{
	style = Matlab-editor,
	basicstyle=\scriptsize,
	numberstyle=\tiny\ttfamily\color{gray75},
	}
	
\lstdefinestyle{Latex}
	{
	language=[LaTeX]{Tex},
    basicstyle=\scriptsize,
    literate={\$}{{{\bfseries\$}}}1,
    alsoletter={\\,*,\&},
    emph =[1]{\\begin,\\end,\\caption,\\label,\\centering,\\FloatBarrier,
              \\lstinputlisting,\\scalefont,\\addplot,\\input,
              \\legend,\\item,\\subitem,\\includegraphics,\\textwidth,
              \\section,\\subsection,\\subsubsection,\\paragraph,
              \\cite,\\citet,\\citep,\\gls,\\bibliographystyle,\\url,
              \\citet*,\\citep*,\\todo,\\missingfigure,\\footnote},
  	emphstyle =[1]\bfseries,
  	emph = [2]{equation,subequations,eqnarray,figure,subfigure,
  			   condiciones,flalign,tikzpicture,axis,lstlisting,
  			   itemize,description
  			   },
  	emphstyle =[2]\bfseries,
    numbers=none,
	}
	
\lstdefinestyle{Latex-color}
	{
	language=[LaTeX]{Tex},
    basicstyle=\scriptsize,
    commentstyle=\color{dkgreen},
    identifierstyle=\color{black},
    literate={\$}{{{\bfseries\color{Dandelion}\$}}}1, % Colorea el simbolo dollar
    alsoletter={\\,*,\&},
    emph =[1]{\\begin,\\end,\\caption,\\label,\\centering,\\FloatBarrier,
              \\lstinputlisting,\\scalefont,\\addplot,\\input,
              \\legend,\\item,\\subitem,\\includegraphics,\\textwidth,
              \\section,\\subsection,\\subsubsection,\\paragraph,
              \\cite,\\citet,\\citep,\\gls,\\bibliographystyle,\\url,
              \\citet*,\\citep*,\\todo,\\missingfigure,\\footnote},
  	emphstyle =[1]\bfseries\color{RoyalBlue},
  	emph = [2]{equation,subequations,eqnarray,figure,subfigure,
  			   condiciones,flalign,tikzpicture,axis,lstlisting,
  			   itemize,description
  			   },
  	emphstyle =[2]\bfseries,
    numbers=none,
	}
\lstdefinestyle{Java}
{
	basicstyle=\scriptsize,
	language=Java,
}

\lstdefinestyle{Java-color}
{
	basicstyle=\scriptsize,
	language=Java,
  	keywordstyle=\color{blue},
  	commentstyle=\color{dkgreen},
  	stringstyle=\color{mauve},
}
\lstdefinestyle{Python}
{
	language=Python,
	basicstyle=\scriptsize,
	otherkeywords={self},  
	keywordstyle=\bfseries,     
	emphstyle=\bfseries,    
	emph={MyClass,__init__},         
}

\lstdefinestyle{Python-color}
{
	language=Python,
	basicstyle=\scriptsize,
	otherkeywords={self},          
	keywordstyle=\bfseries\color{deepblue},
	emph={MyClass,__init__},         
	emphstyle=\bfseries\color{deepred},    
	stringstyle=\color{deepgreen},
}
\lstdefinestyle{R}
{
	language=R,                     
  	basicstyle=\scriptsize,
  	keywordstyle=\bfseries, 
}
\lstdefinestyle{R-color}
{
	language=R,                     
  	basicstyle=\scriptsize,
  	keywordstyle=\bfseries\color{RoyalBlue}, 
  	commentstyle=\color{YellowGreen},
  	stringstyle=\color{ForestGreen}  
}




%%%%%%%%%%%%%%%%%%%%%%%%%%
% Frase celebre
%%%%%%%%%%%%%%%%%%%%%%%%%%
\newenvironment{FraseCelebre}
{\begin{list}{}{%
      \setlength{\leftmargin}{0.5\textwidth}
      % Desplazamos el inicio de
      % los párrafos a la derecha la mitad
      % de la anchura de la línea de texto.
      % Puede que quieras cambiar esto
      % por otra cantidad como '5cm'.
      \setlength{\parsep}{0cm}
      % La separación entre párrafos de la
      % frase o de la fuente es normal, sin
      % espacio extra.
      \addtolength{\topsep}{0.5cm}
      % Aumentamos un poco la separación
      % entre la parte de la fase célebre
      % y los párrafos de alrededor
    }
  }
  {\unskip \end{list}}

\newenvironment{Frase}%
{\item \begin{flushright}\small\em}%
  {\end{flushright}}

\newenvironment{Fuente}%
{\item \begin{flushright}\small}%
  {\end{flushright}}

\newenvironment{bottomparagraph}{\par\vspace*{\fill}}{\clearpage}

% Para que las páginas en blanco no tengan numeración.
\newcommand{\clearemptydoublepage}{\newpage{\pagestyle{empty}\cleardoublepage}}

%%%%%%%%%%%%%%%%%%%%%%%%%%%%
% Configuración del español extra
%%%%%%%%%%%%%%%%%%%%%%%%%%%%
\addto\captionsspanish{%
  \renewcommand{\listtablename}{Índice de tablas}
  \renewcommand{\tablename}{Tabla}
  \renewcommand{\lstlistingname}{Código}
  \renewcommand{\lstlistlistingname}{Índice de Códigos}
  \renewcommand{\glossaryname}{Glosario}
  \renewcommand{\acronymname}{Acrónimos}
  \renewcommand{\bibname}{Bibliografía}
}

%%%%%%%%%%%%%%%%%%%%%%%%%%%%%%
% Encabezados y demás
%%%%%%%%%%%%%%%%%%%%%%%%%%%%%%
\usepackage{fancyhdr}
\pagestyle{fancy}

\fancyhf{} % limpia encabezado y pie

% Encabezado izquierdo (páginas pares): capítulo
\fancyhead[LE]{\scshape \leftmark}

% Encabezado derecho (páginas impares): sección
\fancyhead[RO]{\scshape \rightmark}

% Línea horizontal en el encabezado
\renewcommand{\headrulewidth}{0.2pt}

% Número de página centrado en el pie
\fancyfoot[C]{\thepage}

% Redefinir \chaptermark para que \leftmark muestre "Capítulo N. Título"
\renewcommand{\chaptermark}[1]{%
  \markboth{\chaptername\ \thechapter.\ #1}{}}

% Redefinir \sectionmark para que \rightmark muestre "N.M Sección"
\renewcommand{\sectionmark}[1]{%
  \markright{\thesection\ #1}}
